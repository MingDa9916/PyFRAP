\documentclass[a4paper,10pt,twocolumn]{article}
\usepackage[utf8]{inputenc}


%Misc
\usepackage[english]{babel}
\usepackage{hyperref}


%Bibliography
\usepackage[round,comma,sort,authoryear]{natbib}



\newcommand{\nref}{\textbf{(need ref!)}}
\newcommand{\invivo}{\textit{in vivo }}
\newcommand{\invitro}{\textit{in vitro }}



%opening
\title{Going beyond simplified models: Improving FRAP analysis using PyFRAP}
\author{Alexander Bl\"a\ss le, Gary Soh, Patrick M\"uller}


\begin{document}

\maketitle

\begin{abstract}
The transport of macromolecules is crucial for development and homeostasis and must be tightly regulated to prevent developmental defects and disease states. Macromolecules such as proteins must often move 
through complex environments from their site of production to their targets. What are the biophysical properties of such molecules, and how is their transport affected by the properties of the environment 
through which they move? One technique to address these questions is FRAP (Fluorescence Recovery After Photobleaching), which measures the mobility-driven recovery of fluorescent molecules in a photobleached 
area. Although FRAP assays have been used for 40 years, there are several shortcomings of current data analysis methods that preclude the retrieval of quantitative information from FRAP experiments. 
Most analysis algorithms make simplified assumptions about the sample geometry, bleaching conditions, and the underlying kinetics of the system that do not fully reflect the complexities of the experimental 
sample. Furthermore, the algorithms used to analyze FRAP data cannot always be easily implemented in laboratories without the appropriate computing background. To address these shortcomings, we developed the
Python software package “PyFRAP”. PyFRAP numerically solves the equations that govern macromolecule transport on an exact three-dimensional model of the sample using finite element simulations. To account for
bleaching inhomogeneities, PyFRAP matches the FRAP image data with the initial condition on the simulation. Simulated solutions are then fitted to the experimental data using a variety of optimization 
algorithms. PyFRAP comes with an intuitive graphical user interface that can be applied to a variety of systems. Moreover, we present protocols for experimental setups ranging from custom in vitro assays 
involving only pure diffusion to in vivo assays involving complex geometries, tortuosity and reaction-kinetics in zebrafish embryos  and show how such assays can be used to compose complex FRAP experiments 
step-by-step. These assays in combination with PyFRAP allow to selectively measure the contribution of factors such as sample geometry and reaction-kinetics to FRAP analysis results and might unravel how 
global diffusion rates are established. We present in vitro results for fluophore-dextrans ranging over two orders of magnitude in molecular weight and compare the resulting diffusion coefficients with
literature values, showing that PyFRAP is able to report diffusion coefficents over multiple time scales.  Through inserting polyacrylamide beads into the in vitro assays, we are able to report that tortuosity
can decrease global diffusion rates up to twofold for a multitude of macromolecules.  We show that tortuosity is similar in vivo and in vitro for recombinant GFP by performing FRAP experiments in living 
zebrafish embryos. By repeating the same in vivo experiments with secreted GFP, we display that PyFRAP is able to account for underlying kinetics such as production. Finally, we measure the diffusion rate of 
squint-GFP, a TGF$\beta$protein that is crucial for germ-layer patterning, to illustrate that  extracellular binding might decrease FRAP diffusion rate estimates up to five-fold.
\end{abstract}

\section{Introduction}

Macromolecules control various mechanisms in development and homeostasis. To achieve this, they often need to be transported from their source location to their target. For example, the Nodal protein Squint,
a key player in the determination of germ-layer formation and a member of the TGF$\beta$ superfamily is secreted at the marignal zone, however reaches deep into the tissue during zebrafish development 
\citep{Mueller2012}. Most transport mechanisms are suspected to be regulated by the macromolecule's biophysical properties, such as its stabiliy \nref and diffusion rate \nref. In 1974 both Poo and Cone and
Liebman and Entine simultanrously describe a method for measuring diffusion rates of fluorescent macromolecules: During Fluorescence Recovery After Photobleaching (FRAP) assays the fluorescence of
molecules in a given region in the sample is bleached by exposure to a strong laser pulse. When the region is sufficiently bleached, one monitors the recovery of the fluorophore intensity inside the region
until full recovery. If diffusion happens at a faster time-scale than other kinetics, the observed monitored recovery can then be used to compute the diffusion rate of the fluorescent macromolecule.
The computation of diffusion rates is mostly done by fitting curves computed from theoretical solutions to the measured recovery curves \nref. However, such theoretical solutions rely on a set of assumptions that are not necessarily 
fullfilled:\\
Most current analysis methods reduce the FRAP analysis to an either one-dimensional or two-dimensional problem, often ignoring that biological samples have more complex geometries that could play
important roles in molecule movement \nref. For instance, during zebrafish embryogenesis, the germ layers are defined starting at dome stage. Dome stage is characterized by a roughly 15 cell thick layer on 
top of the embryo's yolk sack. While a FRAP recovery curve of such a sample is dominated by the concentration close to the location of imaging, concentrations close to the yolk will contribute later to the
recovery. \\
Especially in \textit{in vivo} experiments, the investigated fluorescent molecules are generated by injected mRNA and degrade over time. Since fluorecence recovery in such samples often takes up to an hour,
the production and degradation can contribute to the observed recovery curve, however are mostly neglected in classical FRAP analys is models. \\
Moreover, the bleaching process in FRAP experiments is often modeled too ideal: The bleaching is assumed to be equal throughout the bleached ROI, while the molecules outside of the bleached ROI stay unbleached.
However, if the sample is inhomogeneous due for example having both intracellular and extracellular regions or if there is a delay between bleaching and the time point at which the experiment is started,
the initial image of the recovery series shows inhomogeneities both inside and outside of the bleached ROI.\\
In this publication we present PyFRAP. PyFRAP is a versatile Python-based FRAP analysis software that accounts for the previously mentioned drawbacks of current analysis methods: Instead of fitting a curve 
derived from a theoretical model, PyFRAP simulates the actual FRAP experiment using a finite element scheme and then fits the resulting curve to the experiment data. This allows to tackle all drawbacks at the 
same time: PyFRAP simulates the FRAP experiment in a three-dimensional geometry, using an interpolation of the initial post-bleaching image as initial conditions, solving the problem of geometry and realistic
bleaching conditions. During the fitting process, PyFRAP is able add both production and degradation to the solution of the simulated FRAP experiment. \\
To facilitate FRAP analysis, PyFRAP comes with an easy-to-handle graphical user interface (GUI). It allows users even without any computational background to access the whole PyFRAP suite, ranging from 
different image manipulation methods to improve image analysis to statistical model comparison methods. Moreover, PyFRAP is completely open source, comes with a fully documented API, making it easily 
customizable to the user's needs and sharing it on Github allows rapid growth of expansion of the PyFRAP software suite through collaborative work.
Together with PyFRAP, we present a series of both \invitro and \invivo FRAP protocols. To benchmark our FRAP analysis, we performed FRAP experiments for fluophore-dextrans ranging from 3kDa dextrans to 500kDa 
dextrans and recombinant GFP, covering multiple orders of magnitude. Our predicted diffusion rates are overall  in agreement with literature values, showing that our \invitro assay together with PyFRAP deliver robust diffusion 
estimates over multiple time-scales. In more realistic setups, the diffusing molecule often has to overcome obstacles such as cells, leading to a tortuous movement. To investage the influence of such obstacles,
we performed similar \invitro experiments with polyacrylamide beads mixed into the sample for recombinant GFP and multiple fluophore-dextrans. We found that tortuosity can decrease global diffusion rates up 
to twofold. The transition from \invitro to \invivo is done by measuring global diffusivity of injected recombinant GFP in living zebrafish embryos at dome stage. Our results indicate that GFP diffusivity \invivo
are comparable to our \invitro results with beads. By injecting GFP mRNA instead of the recombinant protein into the zebrafish embryo, we add production to our system. Due to PyFRAP's ability to incorporate
reaction kinetics, we are able to measure the same diffusivity compared to the \invivo experiments with recombinant GFP, showing that our software is able to successfully account for simple reaction kinetics
without skewing diffusion rate estimates. Extracellular binding is supposed to limit diffusion and accountable for differences in macromolecule diffusivity \citep{Mueller2013}. To test by how much 
extracellular binding can reduce GFP diffusivity, we injected the Nodal protein Squint-GFP into zebrafish embryos and measured its diffusivity. We are able to reproduce diffusion rates as reported in
\citep{Mueller2012}, showing that extracellular binding in this case lowers the diffusion rate five-fold.\\
The set of \invitro and \invivo assays presented in this publication in combination with PyFRAP shows how FRAP measurements can be simplified to dissect complicated FRAP measurements to identify the contribution
of single factors to the measured global diffusion coefficent.

\section{Methods}

\subsection{PyFRAP}

PyFRAP is an open-source Python-based FRAP analysis software that runs on all major operating systems Microsoft Windows, Mac OSX and Linux. PyFRAP comes with an intuitive GUI, a fully documented API allowing 
quick creation of scripts or modifications of the PyFRAP code, manifold of image processing functions specificly useful for FRAP analysis, arbitray geometry and ROI definition, a finite element PDE solver 
that simulates FRAP experiments with various options, statistical tools for results comparison, averaging and model comparison and multiple plotting and I/O functions. A FRAP analysis with PyFRAP can be 
divided into 4 subtasks:

\begin{itemize}
 \item Defining experiment conditions,
 \item running the image analysis,
 \item simulation the FRAP experiment, and
 \item fitting the simulation to the analyzed data.
\end{itemize}

We will now describe these steps more in detail.

\subsubsection{Experiment definition}

To achieve the best FRAP analysis results, it is important to define the experiment as exactly as possible.


\subsubsection{Simulation}






\section{Results}

\section{Discussion}






\end{document}
